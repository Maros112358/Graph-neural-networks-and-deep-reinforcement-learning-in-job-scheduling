
\chapwithtoc{Conclusion}

In this thesis, we explored the application of graph neural networks and deep reinforcement learning to job scheduling. We discussed different architectures and algorithms for learning PDRs. We presented five models published in the literature with source code available on GitHub. We also extended a subset of these models to the dynamic variant of job-shop scheduling. We then experimentally compared these models on three different variants of job scheduling: JSSP, FJSP, and DJSP. From our experiments, we observed that the selection of input features had a much more significant impact on model performance than the architecture of the graph neural network. In JSSP, the remaining work seemed to be the most crucial feature. In FJSP, processing time appeared to be the most important feature. In DJSP, the lower bound of the estimated completion time seemed more influential for sparsely incoming jobs. For more densely incoming jobs, the remaining work seemed more practical. Investigating the effect of existing input features and searching for new ones may be a valuable direction for future work.


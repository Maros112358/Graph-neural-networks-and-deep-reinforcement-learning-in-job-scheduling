
\chapwithtoc{Introduction}

Job scheduling considers the optimal allocation of a number of jobs to a set of shared resources. Each job consists of multiple operations with predefined constraints (e.g., operations are processed in a specific order). The goal is to minimize total makespan, tardiness, cost, or other objectives \cite{YamadaNakanoJSSP}. Each resource can process only one operation at a time and is non-preemptive, i.e., it runs without interruption.
\par
Different constraints and conditions lead to different variants of job scheduling. 
Generally, the Job shop scheduling problem assumes all jobs to be known apriori and that each operation can be allocated only to one given machine \cite{YamadaNakanoJSSP}. In flexible job-shop scheduling, each operation can be allocated to any of the machines from a given subset of machines \cite{DAUZEREPERES2024409}. Dynamic job-shop scheduling, one of the more common variants at the time, tackles the stochastic aspects of modern manufacturing, e.g., the arrival of new jobs during the execution of the schedule or uncertain processing time \cite{MOHAN201934}.
\par
Due to the NP-hardness of these problems \cite{Garey1976TheCO}, numerous approaches and heuristics have been proposed over time to yield approximate solutions \cite{Jansen2000ApproximationAF}. Extensively studied meta-heuristics for job scheduling include genetic algorithms \cite{PEZZELLA20083202, zhang2011effective}, tabu search \cite{Brandimarte_1993}, memetic algorithms \cite{frutos2010memetic}, particle swarm optimization \cite{ZHANG20091309}, and simulated annealing \cite{Yamada1996}.
\par
Priority dispatching rules (PDRs) \cite{Haupt1989ASO} are a computationally fast and intuitive heuristic method compared to other optimization methods and is widely used in scheduling systems. Many different PDRs have been proposed and extensively studied in the literature \cite{doi:10.1080/00207543.2011.611539}. Designing a quality PDR is usually a very time-consuming task requiring extensive domain knowledge. Deep reinforcement learning has already been proposed as a possible solution for the automatization of algorithm learning for combinatorial optimization problems \cite{bengio2020machine}. Several recent works have focused on extending this technique to job scheduling \cite{zhang2020learning, https://doi.org/10.1002/tee.23788, DBLP:journals/corr/abs-2106-01086, 10114974, 9826438, 10226873} applying graph neural networks on a graph representation of job scheduling problems \cite{BLAZEWICZ2000317}.\\
\par
This presents the overview of five publicly available models for solving job shop scheduling using graph neural networks and deep reinforcement learning. In \textbf{Chapter 1}, we define job shop scheduling problems and their variants. We also describe priority dispatching rules, how they are used in the context of job shop scheduling, and how their design can be automated using reinforcement learning. In \textbf{Chapter 2}, we introduce the concept of graph neural networks and deep reinforcement learning. We discuss different neural architectures and training algorithms. In \textbf{Chapter 3}, we present five publicly available models found in the literature and on GitHub. In \textbf{Chapter 4}, we experimentally compare these models and present the results. In \textbf{Chapter 5}, we interpret the results of our experiments.
\chapter{Discussion}

This chapter will discuss and interpret our findings from the previous chapter.

\section{JSSP} \label{discussion_jssp}
As can be seen from table 4.1 and figures 4.1, 4.2, 4.3, and 4.4 in \ref{results_jssp}, the best performance model was achieved by model \textbf{IEEE-ICCE-RL-JSP}, followed by model \textbf{fjsp-drl} and priority dispatching rule \textbf{MWKR}. We would like to point out that the results of \textbf{MWKR} PDR performance we obtained are much better than those reported by authors of \textbf{L2D} in their paper \cite{zhang2020learning}. However, their MWKR PDR's source code is unavailable, so we could not investigate the cause of this difference.
\par
Good performance of \textbf{MWKR} PDRs hints at the importance of the sum of processing times of remaining operations as an operation feature. Models \textbf{L2D} and \textbf{End-to-end-DRL-for-FJSP} did not include this feature. We also did not select this as a feature for \textbf{Wheatley}. Since \textbf{Wheatley} has easily customizable operation features, the idea for further research would be to add this operation feature to \textbf{Wheatley}. On the other hand, model \textbf{IEEE-ICCE-RL-JSP} used the number of remaining operations in the current job as a feature, as discussed in \ref{model_iee_icce_rl_jsp}. Model \textbf{fjsp-drl} also included the number of unscheduled operations in the job as a feature, as discussed in \ref{model_fjsp_drl}. This further points to the importance of the remaining work as a feature. 

\section{FJSP}
As can be seen from Table 4.6 in \ref{results_fjsp}, the best results were obtained by model \textbf{End-to-end-DRL-for-FJSP}. Results are also statistically significant. Both models performed much better than baseline PDRs. Tables 4.6 and Figures 4.7 show that the machine selection PDR \textbf{SPT} improves the makespan significantly. This hints at the importance of the processing time as a feature in the model. The model \textbf{End-to-end-DRL-for-FJSP} includes processing time as a feature explicitly in the machine embeddings. Model \textbf{fjsp-drl} doesn't include processing time as a machine feature. On the other hand, it includes the time when the machine will finish all its currently assigned operations as a feature, as discussed in \ref{model_fjsp_drl}, which is similar to machine selection PDR \textbf{EET}. This may explain why the model \textbf{fjsp-drl} yielded worse results. It would be interesting to see the effect of this feature on the performance of PDRs in future research.

\section{DJSP}
For load factor 1, \textbf{L2D} yielded the best results, as can be seen in Table 4.9. For load factor 2, \textbf{IEEE-ICCE-RL-JSP} and \textbf{L2D} produced similar makespans. \textbf{L2D} worked better for larger instances. For load factor 4, \textbf{IEEE-ICCE-RL-JSP} yielded the best results. 
\par
Worse results yielded by \textbf{Wheatley} for all load factors can result from the poor performance in solving static JSSP.
\par
A lower load factor equates to a greater time between the arrival of two consecutive jobs. In \textbf{L2D}, the initial node features include boolean $I(o)$ representing if operation $o \in O$ has already been scheduled and the lower bound of the estimated time of completion $C_{LB}(o)$. Then, the result of this experiment can be interpreted that for more sparsely arriving jobs, $C_{LB}(o)$ is a more important feature. As the load factor increases, the DJSP is more similar to static JSSP, and other features, which are more important for JSSP, e.g., the sum of processing times of remaining operations discussed in \ref{discussion_jssp}, become more critical.

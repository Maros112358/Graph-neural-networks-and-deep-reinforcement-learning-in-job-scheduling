

%%% Choose a language %%%

\newif\ifEN
\ENtrue   % uncomment this for english
%\ENfalse   % uncomment this for czech

%%% Configuration of the title page %%%

\def\ThesisTitleStyle{mff} % MFF style
%\def\ThesisTitleStyle{cuni} % uncomment for old-style with cuni.cz logo
%\def\ThesisTitleStyle{natur} % uncomment for nature faculty logo

\def\UKFaculty{Faculty of Mathematics and Physics}
%\def\UKFaculty{Faculty of Science}

\def\UKName{Charles University in Prague} % this is not used in the "mff" style

% Thesis type names, as used in several places in the title
% \def\ThesisTypeTitle{\ifEN BACHELOR THESIS \else BAKALÁŘSKÁ PRÁCE \fi}
\def\ThesisTypeTitle{\ifEN MASTER THESIS \else DIPLOMOVÁ PRÁCE \fi}
%\def\ThesisTypeTitle{\ifEN RIGOROUS THESIS \else RIGORÓZNÍ PRÁCE \fi}
%\def\ThesisTypeTitle{\ifEN DOCTORAL THESIS \else DISERTAČNÍ PRÁCE \fi}
% \def\ThesisGenitive{\ifEN bachelor \else bakalářské \fi}
\def\ThesisGenitive{\ifEN master \else diplomové \fi}
% \def\ThesisGenitive{\ifEN rigorous \else rigorózní \fi}
%\def\ThesisGenitive{\ifEN doctoral \else disertační \fi}
% \def\ThesisAccusative{\ifEN bachelor \else bakalářskou \fi}
\def\ThesisAccusative{\ifEN master \else diplomovou \fi}
%\def\ThesisAccusative{\ifEN rigorous \else rigorózní \fi}
%\def\ThesisAccusative{\ifEN doctoral \else disertační \fi}



%%% Fill in your details %%%

% (Note: \xxx is a "ToDo label" which makes the unfilled visible. Remove it.)
\def\ThesisTitle{Graph neural networks and deep reinforcement learning in job scheduling}
\def\ThesisAuthor{Maroš Bratko}
\def\YearSubmitted{2024}

% department assigned to the thesis
\def\Department{Department of Theoretical Computer Science and Mathematical Logic}
% Is it a department (katedra), or an institute (ústav)?
\def\DeptType{Department}

\def\Supervisor{prof. RNDr. Ing. Martin Holeňa, CSc.}
\def\SupervisorsDepartment{Department of Theoretical Computer Science and Mathematical Logic}

% Study programme and specialization
\def\StudyProgramme{Computer Science - Artificial Intelligence (N0619A140003)}
\def\StudyBranch{IUIP (0619TA140003)}

\def\Dedication{%
Dedication. \xxx{It is nice to say thanks to supervisors, friends, family, book authors and food providers.}
}

\def\AbstractEN{%
The priority dispatching rule (PDR) is a greedy heuristic algorithm used to obtain approximate solutions to the NP-hard job scheduling problem (JSP). The manual design of PDRs requires domain knowledge to achieve good results. Recently, deep reinforcement learning has been used to automate the process of designing PDRs, where PDRs are formulated as a Markov Decision Process exploiting graph representation of JSP, and a graph neural network (GNN) selects the operations to be dispatched. In this thesis, we present the overview of five models published in literature with their source code publicly available on GitHub and experimentally compare their performance on three different variants of JSP. Our experiments show that the choice of input features, e.g., the amount of remaining work, significantly affects the model's performance regardless of the GNN architecture. This suggests that the feature selection is essential for learning high-quality PDRs.
}

\def\AbstractCS{%
Prioritní rozhodovací pravidlo (PDR) je hladový heuristický algoritmus používaný ke získaní přibližného řešení NP-těžkého úkolu organizace práce (JSP). Manuální příprava PDR vyžaduje zkušenosti v dané oblasti pro získaní dobrých výsledků. V posledních letech bylo k automatizaci přípravy PDR využito zpětnovazební učení, kde jsou PDR formulováný jako Markovův rozhodovací proces využívajíci grafovou reprezentaci JSP, ve které grafová neurónová síť (GNN) vybírá operace ke spuštění. V této prácí uvádíme přehled pěti modelů publikovaných v literatuře, kterých zdrojové kódy jsou volně dostupné na GitHubu, a experimentálňe porovnáme jejich výsledky. Naše experimentuji ukazují, že volba vstupních příznaků, napr. množství zbývající práce, významně ovlivňuje kvalitu výsledných řešení bez ohledu na architekturu GNN. To naznačuje, že výběr vstupních příznaků je pro přípravu kvalitních PDR zásadní.
}

% 3 to 5 keywords (recommended), each enclosed in curly braces.
% Keywords are useful for indexing and searching for the theses by topic.
\def\Keywords{%
{job-shop scheduling} {graph neural networks} {deep reinforcement learning} {markov decision process}
}

% If your abstracts are long and do not fit in the infopage, you can make the
% fonts a bit smaller by this setting. (Also, you should try to compress your abstract more.)
% Alternatively, consider increasing the size of the page by uncommenting the
% geometry modification in thesis.tex.
\def\InfoPageFont{}
%\def\InfoPageFont{\small}  %uncomment to decrease font size

\ifEN\relax\else
% If you are writing a czech thesis, you additionally need to fill in the
% english translation of the metadata here!
\def\ThesisTitleEN{\xxx{Thesis title in English}}
\def\DepartmentEN{\xxx{Name of the department in English}}
\def\DeptTypeEN{\xxx{Department}}
\def\SupervisorsDepartmentEN{\xxx{Superdepartment}}
\def\StudyProgrammeEN{\xxx{study programme}}
\def\StudyBranchEN{\xxx{study branch}}
\def\KeywordsEN{%
\xxx{{key} {words}}
}
\fi
